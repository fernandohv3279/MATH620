\documentclass{article}
\usepackage{amsmath}
\usepackage{amsthm}
\usepackage{amsfonts}
\usepackage{mathtools}
\title{MATH 620: Homework 2}
\author{Fernando}
\date{October 10, 2023}
\begin{document}
\maketitle
\section{Problem 1}
\subsection{Part a}
We want to write $D=\{x_0+y: y\in V\}$ (where V is a linear subspace).
In this case we can take $x_0=3x+2$ and $V=\{\phi\in X:
\phi(0)=\frac{\partial\phi}{\partial x}(1)=0\}$. It
is easy to see that $D$ is a linear subspace because it is closed under adition
and scalar product. Also the boundary conditions are satisfied so indeed we can
write $D$ in this way.
\subsection{Part b}
Take $\phi_1,\phi_2\in D$ we have to prove that:
$(1-\alpha)\phi_1+\alpha \phi_2 \in D,\forall \alpha \in [0,1]$.
This is clearly in $C^k([0,1])$ but also:
\[
\left((1-\alpha)\phi_1+\alpha\phi_2\right)(0)
=\left((1-\alpha)2+\alpha2\right)=2,
\]
and
\[
\left[\frac{\partial}{\partial x}\left((1-\alpha)\phi_1+\alpha \phi_2\right)\right](1)= 
\left((1-\alpha)\frac{\partial \phi_1}{\partial x}+\alpha\frac{\partial
\phi_2}{\partial x}\right)(1) = (1-\alpha)3+\alpha 3=3.
\]
So $D$ is convex.
\subsection{Part c}
\section{Problem 2}
\subsection{Part a}
\subsection{Part b}
\section{Problem 3}
\section{Problem 4}
\section{Problem 5}
\end{document}
