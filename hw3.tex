\documentclass{article}
\usepackage{amsmath}
\usepackage{amsthm}
\usepackage{amsfonts}
\usepackage{mathtools}
\usepackage{listings}
\title{MATH 620: Homework 3}
\author{Fernando}
\date{October 17, 2023}
\begin{document}
\maketitle
\section{Problem 1}
\subsection{Part a}
In this case
\[
	f(x,u,\xi)=\mu(\sqrt{1+(u')^2}-1)-gu,
\]
so
\[ f_u=-g \]
\[ f_\xi =\frac{\mu u'}{\sqrt{1+(u')^2}} \]
\[ \frac{d}{dx} f_\xi =\frac{\mu u''}{(1+(u')^2)^{3/2}}. \]
Then the E-L equation is
\[
\frac{\mu u''}{(1+(u')^2)^{3/2}} = -g,
\]
or equivalently:
\[
\mu u'' -g\cdot(1+(u')^2)^{3/2} = 0,
\]
with initial conditions $u(a)=u(b)=1$.
\subsection{Part b}
In order to prove that $\overline u$ is a minimizer it is enough to show that
the map $(x,\xi)\mapsto f(x,u,\xi)$ is convex for every $x$.
In other words: is the function
\[
	h((x,y))=\mu(\sqrt{1+y^2} -1) -gx
\]
convex?

The answer is yes. We can see that this function is linear in $x$ and for $y$
we have that
\[
	\partial_{yy} h = \frac{\mu}{(1+y^2)^{3/2}}.
\]
In this particular case this is enough to conclude that the function is convex.
Because it is convex in the $y$ direction and linear in the $x$ direction. This
is an intuitive explanation, if we want to be more precise we can compute the
Hessian matrix but we will get the same result.

\textbf{Uniqueness?}

TODO
\section{Problem 2}
\subsection{Part a}
\subsection{Part b}
\subsection{Part c}
\section{Problem 3}
\subsection{Part a}
\subsection{Part b}
\section{Problem 4}
\section{Problem 5}
% LEAL
\end{document}
