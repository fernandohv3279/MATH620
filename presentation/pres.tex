\documentclass{beamer}

\usetheme{simple}

% \usepackage{lmodern}
% \usepackage[scale=2]{ccicons}
\usepackage{amsmath}
\usepackage{svg}
\usepackage{ulem}
\usepackage{mathtools}

\title{Material distribution on wheels}
% \subtitle{and the isoperimetric inequality}
\date{\today}
% \author{Fernando Herrera}
% \institute{MATH 620}

\begin{document}

\maketitle

\begin{frame}{The problem}
    \begin{center}
        \begin{itemize}
           \item What is the strongest wheel we can build\\
               using X amount of material?\\
           \pause
           \item What do we mean by strong?\\
           \pause
           \quad Stiff (Pa)
           \pause
           \item What are the properties of the material?\\
           \pause
           \quad Isotropic
        \end{itemize}
    \end{center}
\end{frame}

\begin{frame}{Approach}
    \begin{center}
        \Huge
    \[
        w:D_1 \to \{0,1\}
    \]
    \pause
    \[
        \rho:D_1 \to [0,1]
    \]
    \end{center}
\end{frame}

\begin{frame}{Approach}
    \begin{center}
        \Huge
    \[
        \min_\rho F(\rho)=\int_Df(\rho)
    \]
    Subject to
    \[
        \int_D\rho dV\leq V_0
    \]
    \pause
    State field = ?
    \end{center}
\end{frame}

\begin{frame}{Approach 2}
    \begin{center}
        \Huge
        Bends{\o}e \& Sigmund
    \end{center}
\end{frame}

\end{document}
